% !TEX root = ./First_year_report.tex


This report provides an overview of the work carried out in the first academic year of the PhD project I am embarked on. The main goal of the project is to advance the ways quantum computing can address the computational challenges in High Energy Physics (HEP). In particular, the project hopes to contribute to satisfying simulation needs for near-term neutrino experiments such as DUNE.

Quantum computing is a booming technology field. In the near future, it is expected to lead to advances in encryption, chemical calculations, and pharmaceutical development \cite{Bova2021,Urbanek2020}. The development of quantum computing aims to address the limitations of conventional computers by leveraging the distinctive properties of quantum systems, such as superposition and entanglement. The ability of quantum computers to surpass conventional computers is yet to be beyond discussion. In particular, current hardware is significantly limited, and, due to this, it does not always provide results a classical computer could not achieve or simulate. It is known, however, that there are quantum states that cannot be replicated classically, and could form the basis of quantum advantage\cite{Oliviero2022,Zhang2024}. Furthermore, some quantum algorithms have been found to provide exponential speedups relative to classical computation. In the case of intrinsically quantum Particle Physics, powerups could also stem from analog simulation of lattice systems, for example\cite{Humble2022}.

Efforts to bring forward a new era of quantum computing are focused on the development of scalable Quantum Error Correction \cite{Girvin2021,Devitt2013,Roffe2019}. Quantum Error Correction stands as an important field of research, as it is essential to developing reliable computing methods able to withstand noise and environmental effects. Without robust error correction strategies, it is not possible to achieve fault tolerance, ensuring a system's ability to function correctly. The exploration being carried out in this area could bring forth the performance speedups of future quantum computing.

Quantum error correction consists of both a theoretical and conceptual framework (as given by codespaces and Kraus representations) and a collection of specific techniques. Existing codes, such as the three-qubit code and surface codes presented in this report vary in their effectiveness at tackling certain error types and their resource efficiency.

Over the elapsed period of this PhD, the literature review on the subject provided a solid foundation on the main concepts of Quantum Error Correction, as well as the thought processes and methods that can be used to design and improve correction codes. Moreover, the hands-on programming to replicate code outcomes provided an understanding of the technicalities of the codes. It also fostered familiarity with the computational tools which will be useful in future research stages.

A distinctive aspect of this project is that it is not limited to one computational approach, such as that based on qubits. Due to the collaboration with the SQMS (Superconducting Quantum Materials and Systems) centre at Fermilab, the work benefits from expertise (and potentially hardware) for bosonic quantum computing \cite{sqms}. Qudit-based systems can encode more information within one element (harmonic oscillator) than can be encoded on a two-level qubit. Due to this, it is particularly suited for High Energy Physics applications that can leverage the increased coherence time and strong correlations between states\cite{kurkcuoglu}.

HEP and quantum computing as research fields are particularly suited to synergetic work. Beyond HEP's demand for complex computation, the technologies developed for HEP experiments often overlap with needs of quantum computing hardware. For example, cryogenics and superconductivity are at the core of particle accelerators and quantum computers. As a result, large HEP institutions have become involved in quantum computing, such as the aforementioned SQMS centre at Fermilab and CERN's Quantum Technology Initiative \cite{Humble2022,cernqti}. 

Quantum computing could be incorporated into HEP computational programs such as lattice gauge simulations. Furthermore, it could be a key piece in accelerating event generator matrix-element calculations. For processes such as neutrino-nucleus scattering, often involving large nuclei, error-corrected quantum computers could provide results of higher precision than could be achieved with classical computers \cite{Humble2022}. 

The work presented in this report has close ties to the GENIE event generator, as many project collaborators are based at the University of Liverpool. Event generators have a very important role in neutrino physics, and their precision requirements must adapt in view of large upcoming experiments\cite{Andreopoulos2021}. Therefore, we hope that the potential advances in quantum computing applications to HEP could be incorporated into this program. Due to this, the focus of the quantum computing work will prioritize work on nuclear dynamics for neutrino-nucleus scattering cross-section estimation.

The report is structured as follows. First, an overview of key concepts in quantum computing is provided, alongside some comments on the current state of the field. Secondly, the topic of Quantum Error Correction is introduced in detail, with some algorithm examples and implementations. This topic will be covered from both qubit and qudit-based perspectives. Then, a brief discussion of the challenges in neutrino physics and neutrino event generation is provided. To conclude, the research questions that will be targeted in future PhD work are outlined.





