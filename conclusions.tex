% !TEX root = ./First_year_report.tex
\chapter{Summary}

This report provided an overview of the basic ideas behind quantum computing, before delving into one of the fundamental aspects of computation. Quantum error correction stands as an important field of research, as it is essential to developing reliable computing methods, able to withstand noise and environmental effects. Without robust error correction strategies, it is not possible to achieve fault tolerance, ensuring a system's ability to function correctly. 

Quantum error correction consists of both a theoretical and conceptual framework (as given by codespaces and Kraus representations) and a collection of codes. These codes, such as the three-qubit code and surface code presented here, can be differentiated by their effectiveness at tackling certain error types, or their efficiency.

Over the elapsed period of this PhD, the literature review on the subject provided a solid foundation on the main concepts of Quantum Error Correction, as well as the thought processes and methods that can be used to design and improve correction codes. Moreover, the hands-on programming to replicate code outcomes provided an understanding of the technicalities of the codes. It also fostered familiarity with the computational tools which will be useful in future research stages.

A distinctive aspect of this project is that it is not limited to one computational approach, such as that based on qubits. Due to the collaboration with the SQMS (Superconducting Quantum Materials and Systems) Center at Fermilab, the work benefits from expertise (and potentially hardware) for bosonic quantum computing. Qudit-based systems can encode more information within one element (harmonic oscillator) than can be encoded on a qubit. Due to this, it is particularly suited for High Energy Physics applications that can leverage the increased coherent time and strong correlations between states.